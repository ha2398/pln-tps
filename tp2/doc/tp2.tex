\documentclass[12pt]{article}

\usepackage{sbc-template}

\usepackage[brazil]{babel}   
%\usepackage[latin1]{inputenc}  
\usepackage[utf8]{inputenc}  
% UTF-8 encoding is recommended by ShareLaTex

\usepackage{hyperref}

\sloppy

\title{Trabalho Prático 2 - Processamento de Linguagem Natural}

\author{Hugo Araujo de Sousa}

\address{
  Processamento de Linguagem Natural (2017/2) \\
  Departamento de Ciência da Computação \\
  Universidade Federal de Minas Gerais (UFMG)
  \email{hugosousa@dcc.ufmg.br}
}

\begin{document} 

\maketitle
     
\begin{resumo} 
  O objetivo desse trabalho é estudar a tarefa de Part-of-Speech (POS) tagging
  para a Língua Portuguesa. Para isso, utilizamos o corpus Mac-Morpho e comparamos
  o desempenho de dois modelos preditivos, um paramétrico e outro não-paramétrico.
\end{resumo}

\section{INTRODUÇÃO}

Em Processamento de Linguagem Natural, uma vez que o principal objetivo é fazer
com que os computadores entendam as linguagens naturais usadas pelos seres
humanos, muitas vezes torna-se necessário reduzir a complexidade dessa tarefa
ao quebrá-la em tarefas intermediárias. Uma dessas tarefas, que é particularmente
útil na área de \textbf{parsing}, é a tarefa de \textbf{Part-of-speech tagging},
que se trata da identificação das classes gramaticais de cada uma das palavras
presentes um corpus \cite{Manning:1999:FSN:311445}. Esse problema não trata-se
apenas da criação de um banco de dados que contenha a classe de cada palavra,
uma vez que uma mesma palavra pode estar associada a múltiplas classes de acordo
com seu contexto e posição em uma sentença.

Uma das abordagens que podem usadas para resolver esse problema é utilizar algum
algoritmo de aprendizado de máquina supervisionado. Nesse trabalho, serão
utilizados dois algoritmos de aprendizado para resolver o problema de POS
tagging a fim de verificar e comparar a precisão e desempenho de cada um.

\section{MODELAGEM}

Dentro dos métodos de aprendizado de máquina supervisionada, existem os 
paramétricos e os não-paramétricos. A diferença entre os dois é que, nos métodos
paramétricos, assume-se que os dados se organizam em algum modelo e então encontra-se
valores apropriados do modelo a partir dos exemplos. Para abordar o problema
de POS tagging, vamos utilizar um algoritmo de cada categoria, sendo
\textbf{Naive Bayes} \cite{McCallum98acomparison} o paramétrico e o classificador
 de \textbf{Support Vector Machines (SVM)} \cite{708428} o não-paramétrico.

Uma vez determinados os algoritmos a serem utilizados no processo de aprendizado,
é necessário discutir o conjunto de dados de entrada e o mapeamento desses dados
para a entrada dos algoritmos.

\subsection{Corpus de Entrada}

O conjunto de dados utilizado no trabalho como entrada dos algoritmos de
aprendizado é o corpus Mac-Morpho \cite{Aluisio2003}. O Mac-Morpho é um corpus
de textos escritos em Português Brasileiro, anotados com as classes gramaticais
de cada palavra presente. Há, disponíveis para download gratuito, as seções
de treinamento, validação e teste do corpus, que representam $ 76\% $, $ 4\% $ e
$ 20\% $ do total do corpus, respectivamente.

Na página do corpus online
\footnote{ \url{http://nilc.icmc.usp.br/macmorpho} } é possível fazer o download
 do mesmo, juntamente com o manual das anotações, que descreve todas as classes gramaticais utilizadas no corpus.

\subsection{Extração de Features}

Com o corpus de entrada em mãos, o próximo passo foi determinar como mapear esses
dados para serem alimentados aos algoritmos de aprendizado de máquina. Para isso,
a decisão foi trabalhar com features das palavras. Essas features são, como o
nome indica, características obtidas através de cada palavra em si, além do seu
contexto na sentença em que se encontra, isto é, posição absoluta e relativa às
classes gramaticais. A lista das features utilizadas no trabalho é mostrada
na Tabela \ref{tab:features}.

\begin{table}[h]
	\centering
	\begin{tabular}{|c|c|}
		\hline
		\textbf{Feature} & \textbf{Descrição} \\ \hline
		\textbf{word} & A própria palavra em si. \\ \hline
		\textbf{is\_first} & Booleano que indica se a palavra é a primeira da sentença. \\ \hline
		\textbf{is\_last} & Booleano que indica se a palavra é a primeira da sentença. \\ \hline
		\textbf{is\_capitalized} & Booleano que indica se a palavra começa com uma letra maiúscula. \\ \hline
		\textbf{is\_all\_caps} & Booleano que indica se a palavra somente contém
		letras maiúsculas. \\ \hline
		\textbf{is\_all\_lower} & Booleano que indica se a palavra contém somente
		letras minúsculas. \\ \hline
		\textbf{prefix-1} & String com o primeiro caractere da palavra. \\ \hline
		\textbf{prefix-2} & String com os dois primeiros caracteres da palavra. \\ \hline
		\textbf{prefix-3} & String com os três primeiros caracteres da palavra. \\ \hline
		\textbf{suffix-1} & String com o último caractere da palavra. \\ \hline
		\textbf{suffix-2} & String com os dois últimos caracteres da palavra. \\ \hline
		\textbf{suffix-3} & String com os três últimos caracteres da palavra. \\ \hline
		\textbf{prev\_tag} & String que representa a classe gramatical da palavra
		anterior à palavra atual. \\ \hline
		\textbf{next\_tag} & String que representa a classe gramatical da palavra
		seguinte à palavra atual. \\ \hline
		\textbf{has\_hyphen} & Booleano que indica se a palavra possui hífen.
		\\ \hline
		\textbf{is\_numeric} & Booleano que indica se a palavra é um número
		(dígitos). \\ \hline
	\end{tabular}
	\caption{\label{tab:features} Features de palavras utilizadas no trabalho.}
\end{table}

\section{IMPLEMENTAÇÃO}

Linguagem
Bibliotecas
Decisões de implementação
Como executar

\section{RESULTADOS}

Saída do programa
Análise

\section{CONCLUSÃO}



\section{REFERÊNCIAS}

\bibliographystyle{sbc}
\bibliography{sbc-template}

\end{document}
