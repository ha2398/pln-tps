\documentclass[12pt]{article}

\usepackage{sbc-template}

\usepackage[brazil]{babel}   
%\usepackage[latin1]{inputenc}  
\usepackage[utf8]{inputenc}  
% UTF-8 encoding is recommended by ShareLaTex

\sloppy

\title{Trabalho Prático 2 - Processamento de Linguagem Natural}

\author{Hugo Araujo de Sousa}

\address{
  Processamento de Linguagem Natural (2017/2) \\
  Departamento de Ciência da Computação \\
  Universidade Federal de Minas Gerais (UFMG)
  \email{hugosousa@dcc.ufmg.br}
}

\begin{document} 

\maketitle
     
\begin{resumo} 
  O objetivo desse trabalho é estudar a tarefa de Part-of-Speech (POS) tagging
  para a Língua Portuguesa. Para isso, utilizamos o corpus Mac-Morpho e comparamos
  o desempenho de dois modelos preditivos, um paramétrico e outro não-paramétrico.
\end{resumo}

\section{INTRODUÇÃO}



\section{MODELAGEM}



\section{IMPLEMENTAÇÃO}



\section{RESULTADOS}



\section{CONCLUSÃO}



\section{REFERÊNCIAS}

\bibliographystyle{sbc}
\bibliography{sbc-template}

\end{document}
