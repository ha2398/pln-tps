\documentclass[12pt]{article}

\usepackage{sbc-template}

\usepackage[brazil]{babel}   
%\usepackage[latin1]{inputenc}  
\usepackage[utf8]{inputenc}  
% UTF-8 encoding is recommended by ShareLaTex

\sloppy

\title{Trabalho Prático 1 - Processamento de Linguagem Natural}

\author{Hugo Araujo de Sousa}

\address{
  Computação Natural (2017/2) \\
  Departamento de Ciência da Computação \\
  Universidade Federal de Minas Gerais (UFMG)
  \email{hugosousa@dcc.ufmg.br}
}

\begin{document} 

\maketitle
     
\begin{resumo} 
  O objetivo desse trabalho é praticar conceitos relacionados a vetores semânticos
  de palavras, trabalhando com esses conceitos em uma aplicação real. Para isso,
  é utilizada uma base real de livros juntamento com uma implmentação do algoritmo
  Skip-Gram.
\end{resumo}

\section{INTRODUÇÃO}


\section{MODELAGEM}


\section{IMPLEMENTAÇÃO}


\section{ESTRUTURA DO PROJETO E EXECUÇÃO}

\subsection{Execução e Parâmetros}


\subsection{Entrada e Saída}


\section{RESULTADOS}


\section{CONCLUSÃO}

\section{REFERÊNCIAS}

\bibliographystyle{sbc}
\bibliography{sbc-template}

\end{document}
